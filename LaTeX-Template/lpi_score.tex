%%
%% This is file `sample-sigconf.tex',
%% generated with the docstrip utility.
%%
%% The original source files were:
%%
%% samples.dtx  (with options: `sigconf')
%% 
%% IMPORTANT NOTICE:
%% 
%% For the copyright see the source file.
%% 
%% Any modified versions of this file must be renamed
%% with new filenames distinct from sample-sigconf.tex.
%% 
%% For distribution of the original source see the terms
%% for copying and modification in the file samples.dtx.
%% 
%% This generated file may be distributed as long as the
%% original source files, as listed above, are part of the
%% same distribution. (The sources need not necessarily be
%% in the same archive or directory.)
%%
%%
%% Commands for TeXCount
%TC:macro \cite [option:text,text]
%TC:macro \citep [option:text,text]
%TC:macro \citet [option:text,text]
%TC:envir table 0 1
%TC:envir table* 0 1
%TC:envir tabular [ignore] word
%TC:envir displaymath 0 word
%TC:envir math 0 word
%TC:envir comment 0 0
%%
%%
%% The first command in your LaTeX source must be the \documentclass
%% command.
%%
%% For submission and review of your manuscript please change the
%% command to \documentclass[manuscript, screen, review]{acmart}.
%%
%% When submitting camera ready or to TAPS, please change the command
%% to \documentclass[sigconf]{acmart} or whichever template is required
%% for your publication.
%%
%%
\documentclass[sigconf]{acmart}

%%
%% \BibTeX command to typeset BibTeX logo in the docs
\AtBeginDocument{%
  \providecommand\BibTeX{{%
    Bib\TeX}}}

%% Rights management information.  This information is sent to you
%% when you complete the rights form.  These commands have SAMPLE
%% values in them; it is your responsibility as an author to replace
%% the commands and values with those provided to you when you
%% complete the rights form.
\setcopyright{acmcopyright}
\copyrightyear{2024}
\acmYear{2024}
\acmDOI{XXXXXXX.XXXXXXX}

%% These commands are for a PROCEEDINGS abstract or paper.
\acmConference[ICoMS 2024]{7th International Conference on Mathematics and Statistics, ICoMS 2024}{June 23--25,
  2024}{Amarante, PT}
%%
%%  Uncomment \acmBooktitle if the title of the proceedings is different
%%  from ``Proceedings of ...''!
%%
%%\acmBooktitle{Woodstock '18: ACM Symposium on Neural Gaze Detection,
%%  June 03--05, 2018, Woodstock, NY}
\acmPrice{15.00}
\acmISBN{978-1-4503-XXXX-X/18/06}


%%
%% Submission ID.
%% Use this when submitting an article to a sponsored event. You'll
%% receive a unique submission ID from the organizers
%% of the event, and this ID should be used as the parameter to this command.
%%\acmSubmissionID{123-A56-BU3}

%%
%% For managing citations, it is recommended to use bibliography
%% files in BibTeX format.
%%
%% You can then either use BibTeX with the ACM-Reference-Format style,
%% or BibLaTeX with the acmnumeric or acmauthoryear sytles, that include
%% support for advanced citation of software artefact from the
%% biblatex-software package, also separately available on CTAN.
%%
%% Look at the sample-*-biblatex.tex files for templates showcasing
%% the biblatex styles.
%%

%%
%% The majority of ACM publications use numbered citations and
%% references.  The command \citestyle{authoryear} switches to the
%% "author year" style.
%%
%% If you are preparing content for an event
%% sponsored by ACM SIGGRAPH, you must use the "author year" style of
%% citations and references.
%% Uncommenting
%% the next command will enable that style.
%%\citestyle{acmauthoryear}


%%
%% end of the preamble, start of the body of the document source.
\begin{document}

%%
%% The "title" command has an optional parameter,
%% allowing the author to define a "short title" to be used in page headers.
\title{International Logistics Performance Feature Extraction Insights and Portugal’s Global Positioning}

%%
%% The "author" command and its associated commands are used to define
%% the authors and their affiliations.
%% Of note is the shared affiliation of the first two authors, and the
%% "authornote" and "authornotemark" commands
%% used to denote shared contribution to the research.
\author{Aldina Correia}
\authornote{Both authors contributed equally to this research.}
\email{aic@estg.ipp.pt}
\orcid{1234-5678-9012}
\authornotemark[1]
\email{webmaster@marysville-ohio.com}
\affiliation{%
  \institution{CIICESI, Escola Superior de Tecnologia e Gestāo, Instituto Politécnico do Porto}
  \streetaddress{Rua do Curral, Casa do Curral, Margaride}
  \postcode{4610-156}
  \city{Felgueiras}
  \country{Portugal}
}

\author{Diogo Ribeiro}
\affiliation{%
  \institution{MySense.ai}
  \streetaddress{7 Bell Yard}
  \postcode{WC2A 2JR}
  \city{London}
  \country{United Kingdom}}
\email{diogo.ribeiro@mysense.ai}


%%
%% By default, the full list of authors will be used in the page
%% headers. Often, this list is too long, and will overlap
%% other information printed in the page headers. This command allows
%% the author to define a more concise list
%% of authors' names for this purpose.
\renewcommand{\shortauthors}{Correia et al.}

%%
%% The abstract is a short summary of the work to be presented in the
%% article.
\begin{abstract}
  The critical role of logistical performance in fostering engineering innovation is universally acknowledged, encompassing the rationalization of supply chains, optimization of inventory management, and facilitating or impeding global collaboration. Efficient logistics integration with innovative technologies is crucial for the prompt delivery of materials and components, enhancing the speed and efficacy of engineering innovation processes. This study examines the robust correlation structure among the Logistics Performance Index (LPI) Indicators over multiple years. The LPI assesses global logistics performance by measuring factors such as the quality of trade and transport infrastructure, the ease of customs procedures, and the efficiency of customs clearance, among other aspects influencing the transnational flow of goods. Our findings confirm the LPI's conceptualization as a latent variable, characterized by its indicators, which demonstrate outstanding internal consistency. This consistency substantiates the LPI's reliability for global logistics performance evaluation. Recognized as a valuable measure of logistical efficiency, the LPI serves as a practical tool in engineering, guiding strategic decision-making and enhancing operational cost-effectiveness and competitiveness.
\end{abstract}

%%
%% The code below is generated by the tool at http://dl.acm.org/ccs.cfm.
%% Please copy and paste the code instead of the example below.
%%
\begin{CCSXML}
<ccs2012>
 <concept>
  <concept_id>10010520.10010553.10010562</concept_id>
  <concept_desc>Computer systems organization~Embedded systems</concept_desc>
  <concept_significance>500</concept_significance>
 </concept>
 <concept>
  <concept_id>10010520.10010575.10010755</concept_id>
  <concept_desc>Computer systems organization~Redundancy</concept_desc>
  <concept_significance>300</concept_significance>
 </concept>
 <concept>
  <concept_id>10010520.10010553.10010554</concept_id>
  <concept_desc>Computer systems organization~Robotics</concept_desc>
  <concept_significance>100</concept_significance>
 </concept>
 <concept>
  <concept_id>10003033.10003083.10003095</concept_id>
  <concept_desc>Networks~Network reliability</concept_desc>
  <concept_significance>100</concept_significance>
 </concept>
</ccs2012>
\end{CCSXML}

\ccsdesc[500]{Computer systems organization~Embedded systems}
\ccsdesc[300]{Computer systems organization~Redundancy}
\ccsdesc{Computer systems organization~Robotics}
\ccsdesc[100]{Networks~Network reliability}

%%
%% Keywords. The author(s) should pick words that accurately describe
%% the work being presented. Separate the keywords with commas.
\keywords{Logistics Performance, LPI, Logistics Decision-Making in Engineering, Feature Aggregation, Exploratory Factor Analysis}
%% A "teaser" image appears between the author and affiliation
%% information and the body of the document, and typically spans the
%% page.

\received{20 February 2007}
\received[revised]{12 March 2009}
\received[accepted]{5 June 2009}

%%
%% This command processes the author and affiliation and title
%% information and builds the first part of the formatted document.
\maketitle

\section{Introduction}
Global logistics efficiency is crucial for companies considering international expansion. To assist with this, the World Bank has developed the Logistics Performance Index (LPI), which evaluates countries based on the quality of their trade infrastructure, ease of customs procedures, and efficiency of goods clearance across borders.

The LPI is derived from surveys conducted with global freight forwarders and logistics professionals, capturing various facets of a country's logistics environment. It measures logistics performance through six key dimensions:

Customs Clearance Process: Evaluates the efficiency of customs and border clearance.
Infrastructure Quality: Assesses the quality of trade and transport infrastructure like ports and roads.
Ease of Arranging Shipments: Looks at the ease of organizing competitively priced shipments.
Competence and Quality of Logistics Services: Rates the skill and quality of logistics providers.
Tracking and Tracing: Measures the ability to track and trace consignments.
Timeliness of Shipments: Checks the regularity with which shipments meet delivery schedules.
Scores range from 1 to 5, with higher scores indicating superior logistics capabilities. This makes the LPI an invaluable tool for governments, businesses, and researchers to compare logistics performance internationally, spot improvement opportunities, and inform policy decisions.

The 2023 LPI edition introduced new Key Performance Indicators (KPIs) based on big data, which include real-time tracking of shipping containers, air cargo, and parcels, providing a broader perspective on global trade dynamics. These new KPIs complement traditional survey data, offering a fuller picture of logistics performance.

Proposals to enhance the LPI suggest various strategic improvements to better measure and optimize logistics conditions, especially important in scenarios where data collection might be compromised, such as during global disruptions.

The LPI thus not only aids businesses in making informed decisions about international operations but also helps countries in developing policies that foster an efficient logistics framework.

The 2023 edition of the International Logistics Performance Index (LPI) enables comparisons across 139 countries, revealing an overall improvement in global logistics performance over the past decade. Notably, the LPI score increased, with a secondary peak emerging around 3.5 between 2018 and 2023, indicating stronger performance among more countries. However, the reduction in sample size from 160 countries in 2018 to 139 in 2023 complicates direct comparisons, especially at the lower score ranges.

The LPI consistently evaluates a broad spectrum of economies, though the specific countries assessed can vary in each edition due to data availability and varying participation levels. For instance, the 2014 edition did not include data for Portugal.

In 2023, the highest LPI scores were predominantly from high-income economies, with Singapore maintaining its top ranking from previous years with a score of 4.3. Eight of the top twelve scorers were European countries. Conversely, the lowest scores generally came from countries with low to lower-middle incomes facing economic challenges from conflicts, natural disasters, or geographic and economic constraints, affecting their integration into global supply chains.

This paper aims to validate the LPI as a reliable measure of logistical performance by analyzing key characteristics and relationships among its indicators. Recognizing the LPI as a robust metric allows it to inform strategic decision-making in engineering and other sectors, improving operational cost-effectiveness and competitiveness. Additionally, the paper tracks Portugal’s LPI performance from 2007 to 2023, focusing on its changes and implications for logistics strategy.

\section{Template Overview}

Some examples.  A paginated journal article \cite{arvis2023}.

\section{Acknowledgments}

Identification of funding sources and other support, and thanks to
individuals and groups that assisted in the research and the
preparation of the work should be included in an acknowledgment
section, which is placed just before the reference section in your
document.

This section has a special environment:
\begin{verbatim}
  \begin{acks}
  ...
  \end{acks}
\end{verbatim}
so that the information contained therein can be more easily collected
during the article metadata extraction phase, and to ensure
consistency in the spelling of the section heading.

Authors should not prepare this section as a numbered or unnumbered {\verb|\section|}; please use the ``{\verb|acks|}'' environment.

\section{Appendices}

If your work needs an appendix, add it before the
``\verb|\end{document}|'' command at the conclusion of your source
document.

Start the appendix with the ``\verb|appendix|'' command:
\begin{verbatim}
  \appendix
\end{verbatim}
and note that in the appendix, sections are lettered, not
numbered. This document has two appendices, demonstrating the section
and subsection identification method.

\section{Multi-language papers}

Papers may be written in languages other than English or include
titles, subtitles, keywords and abstracts in different languages (as a
rule, a paper in a language other than English should include an
English title and an English abstract).  Use \verb|language=...| for
every language used in the paper.  The last language indicated is the
main language of the paper.  For example, a French paper with
additional titles and abstracts in English and German may start with
the following command
\begin{verbatim}
\documentclass[sigconf, language=english, language=german,
               language=french]{acmart}
\end{verbatim}

The title, subtitle, keywords and abstract will be typeset in the main
language of the paper.  The commands \verb|\translatedXXX|, \verb|XXX|
begin title, subtitle and keywords, can be used to set these elements
in the other languages.  The environment \verb|translatedabstract| is
used to set the translation of the abstract.  These commands and
environment have a mandatory first argument: the language of the
second argument.  See \verb|sample-sigconf-i13n.tex| file for examples
of their usage.

\section{SIGCHI Extended Abstracts}

The ``\verb|sigchi-a|'' template style (available only in \LaTeX\ and
not in Word) produces a landscape-orientation formatted article, with
a wide left margin. Three environments are available for use with the
``\verb|sigchi-a|'' template style, and produce formatted output in
the margin:
\begin{description}
\item[\texttt{sidebar}:]  Place formatted text in the margin.
\item[\texttt{marginfigure}:] Place a figure in the margin.
\item[\texttt{margintable}:] Place a table in the margin.
\end{description}

%%
%% The acknowledgments section is defined using the "acks" environment
%% (and NOT an unnumbered section). This ensures the proper
%% identification of the section in the article metadata, and the
%% consistent spelling of the heading.
\begin{acks}
To Robert, for the bagels and explaining CMYK and color spaces.
\end{acks}

%%
%% The next two lines define the bibliography style to be used, and
%% the bibliography file.
\bibliographystyle{ACM-Reference-Format}
\bibliography{sample-base}


%%
%% If your work has an appendix, this is the place to put it.
\appendix

\section{Research Methods}

\subsection{Part One}

Lorem ipsum dolor sit amet, consectetur adipiscing elit. Morbi
malesuada, quam in pulvinar varius, metus nunc fermentum urna, id
sollicitudin purus odio sit amet enim. Aliquam ullamcorper eu ipsum
vel mollis. Curabitur quis dictum nisl. Phasellus vel semper risus, et
lacinia dolor. Integer ultricies commodo sem nec semper.

\subsection{Part Two}

Etiam commodo feugiat nisl pulvinar pellentesque. Etiam auctor sodales
ligula, non varius nibh pulvinar semper. Suspendisse nec lectus non
ipsum convallis congue hendrerit vitae sapien. Donec at laoreet
eros. Vivamus non purus placerat, scelerisque diam eu, cursus
ante. Etiam aliquam tortor auctor efficitur mattis.

\section{Online Resources}

Nam id fermentum dui. Suspendisse sagittis tortor a nulla mollis, in
pulvinar ex pretium. Sed interdum orci quis metus euismod, et sagittis
enim maximus. Vestibulum gravida massa ut felis suscipit
congue. Quisque mattis elit a risus ultrices commodo venenatis eget
dui. Etiam sagittis eleifend elementum.

Nam interdum magna at lectus dignissim, ac dignissim lorem
rhoncus. Maecenas eu arcu ac neque placerat aliquam. Nunc pulvinar
massa et mattis lacinia.

\end{document}
\endinput
%%
%% End of file `sample-sigconf.tex'.

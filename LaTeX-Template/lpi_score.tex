%%
%% This is file `sample-sigconf.tex',
%% generated with the docstrip utility.
%%
%% The original source files were:
%%
%% samples.dtx  (with options: `sigconf')
%% 
%% IMPORTANT NOTICE:
%% 
%% For the copyright see the source file.
%% 
%% Any modified versions of this file must be renamed
%% with new filenames distinct from sample-sigconf.tex.
%% 
%% For distribution of the original source see the terms
%% for copying and modification in the file samples.dtx.
%% 
%% This generated file may be distributed as long as the
%% original source files, as listed above, are part of the
%% same distribution. (The sources need not necessarily be
%% in the same archive or directory.)
%%
%%
%% Commands for TeXCount
%TC:macro \cite [option:text,text]
%TC:macro \citep [option:text,text]
%TC:macro \citet [option:text,text]
%TC:envir table 0 1
%TC:envir table* 0 1
%TC:envir tabular [ignore] word
%TC:envir displaymath 0 word
%TC:envir math 0 word
%TC:envir comment 0 0
%%
%%
%% The first command in your LaTeX source must be the \documentclass
%% command.
%%
%% For submission and review of your manuscript please change the
%% command to \documentclass[manuscript, screen, review]{acmart}.
%%
%% When submitting camera ready or to TAPS, please change the command
%% to \documentclass[sigconf]{acmart} or whichever template is required
%% for your publication.
%%
%%
\documentclass[sigconf]{acmart}

%%
%% \BibTeX command to typeset BibTeX logo in the docs
\AtBeginDocument{%
  \providecommand\BibTeX{{%
    Bib\TeX}}}

%% Rights management information.  This information is sent to you
%% when you complete the rights form.  These commands have SAMPLE
%% values in them; it is your responsibility as an author to replace
%% the commands and values with those provided to you when you
%% complete the rights form.
\setcopyright{acmcopyright}
\copyrightyear{2024}
\acmYear{2024}
\acmDOI{XXXXXXX.XXXXXXX}

%% These commands are for a PROCEEDINGS abstract or paper.
\acmConference[ICoMS 2024]{7th International Conference on Mathematics and Statistics, ICoMS 2024}{June 23--25,
  2024}{Amarante, PT}
%%
%%  Uncomment \acmBooktitle if the title of the proceedings is different
%%  from ``Proceedings of ...''!
%%
%%\acmBooktitle{Woodstock '18: ACM Symposium on Neural Gaze Detection,
%%  June 03--05, 2018, Woodstock, NY}
\acmPrice{15.00}
\acmISBN{978-1-4503-XXXX-X/18/06}


%%
%% Submission ID.
%% Use this when submitting an article to a sponsored event. You'll
%% receive a unique submission ID from the organizers
%% of the event, and this ID should be used as the parameter to this command.
%%\acmSubmissionID{123-A56-BU3}

%%
%% For managing citations, it is recommended to use bibliography
%% files in BibTeX format.
%%
%% You can then either use BibTeX with the ACM-Reference-Format style,
%% or BibLaTeX with the acmnumeric or acmauthoryear sytles, that include
%% support for advanced citation of software artefact from the
%% biblatex-software package, also separately available on CTAN.
%%
%% Look at the sample-*-biblatex.tex files for templates showcasing
%% the biblatex styles.
%%

%%
%% The majority of ACM publications use numbered citations and
%% references.  The command \citestyle{authoryear} switches to the
%% "author year" style.
%%
%% If you are preparing content for an event
%% sponsored by ACM SIGGRAPH, you must use the "author year" style of
%% citations and references.
%% Uncommenting
%% the next command will enable that style.
%%\citestyle{acmauthoryear}


%%
%% end of the preamble, start of the body of the document source.
\begin{document}

%%
%% The "title" command has an optional parameter,
%% allowing the author to define a "short title" to be used in page headers.
\title{International Logistics Performance Feature Extraction Insights and Portugal’s Global Positioning}

%%
%% The "author" command and its associated commands are used to define
%% the authors and their affiliations.
%% Of note is the shared affiliation of the first two authors, and the
%% "authornote" and "authornotemark" commands
%% used to denote shared contribution to the research.
\author{Aldina Correia}
\authornote{Both authors contributed equally to this research.}
\email{aic@estg.ipp.pt}
\orcid{1234-5678-9012}
\authornotemark[1]
\email{webmaster@marysville-ohio.com}
\affiliation{%
  \institution{CIICESI, Escola Superior de Tecnologia e Gestāo, Instituto Politécnico do Porto}
  \streetaddress{Rua do Curral, Casa do Curral, Margaride}
  \postcode{4610-156}
  \city{Felgueiras}
  \country{Portugal}
}

\author{Diogo Ribeiro}
\affiliation{%
  \institution{MySense.ai}
  \streetaddress{7 Bell Yard}
  \postcode{WC2A 2JR}
  \city{London}
  \country{United Kingdom}}
\email{diogo.ribeiro@mysense.ai}


%%
%% By default, the full list of authors will be used in the page
%% headers. Often, this list is too long, and will overlap
%% other information printed in the page headers. This command allows
%% the author to define a more concise list
%% of authors' names for this purpose.
\renewcommand{\shortauthors}{Correia et al.}

%%
%% The abstract is a short summary of the work to be presented in the
%% article.
\begin{abstract}
  The critical role of logistical performance in fostering engineering innovation is universally acknowledged, encompassing the rationalization of supply chains, optimization of inventory management, and facilitating or impeding global collaboration. Efficient logistics integration with innovative technologies is crucial for the prompt delivery of materials and components, enhancing the speed and efficacy of engineering innovation processes. This study examines the robust correlation structure among the Logistics Performance Index (LPI) Indicators over multiple years. The LPI assesses global logistics performance by measuring factors such as the quality of trade and transport infrastructure, the ease of customs procedures, and the efficiency of customs clearance, among other aspects influencing the transnational flow of goods. Our findings confirm the LPI's conceptualization as a latent variable, characterized by its indicators, which demonstrate outstanding internal consistency. This consistency substantiates the LPI's reliability for global logistics performance evaluation. Recognized as a valuable measure of logistical efficiency, the LPI serves as a practical tool in engineering, guiding strategic decision-making and enhancing operational cost-effectiveness and competitiveness.
\end{abstract}

%%
%% The code below is generated by the tool at http://dl.acm.org/ccs.cfm.
%% Please copy and paste the code instead of the example below.
%%
\begin{CCSXML}
<ccs2012>
 <concept>
  <concept_id>10010520.10010553.10010562</concept_id>
  <concept_desc>Computer systems organization~Embedded systems</concept_desc>
  <concept_significance>500</concept_significance>
 </concept>
 <concept>
  <concept_id>10010520.10010575.10010755</concept_id>
  <concept_desc>Computer systems organization~Redundancy</concept_desc>
  <concept_significance>300</concept_significance>
 </concept>
 <concept>
  <concept_id>10010520.10010553.10010554</concept_id>
  <concept_desc>Computer systems organization~Robotics</concept_desc>
  <concept_significance>100</concept_significance>
 </concept>
 <concept>
  <concept_id>10003033.10003083.10003095</concept_id>
  <concept_desc>Networks~Network reliability</concept_desc>
  <concept_significance>100</concept_significance>
 </concept>
</ccs2012>
\end{CCSXML}

\ccsdesc[500]{Computer systems organization~Embedded systems}
\ccsdesc[300]{Computer systems organization~Redundancy}
\ccsdesc{Computer systems organization~Robotics}
\ccsdesc[100]{Networks~Network reliability}

%%
%% Keywords. The author(s) should pick words that accurately describe
%% the work being presented. Separate the keywords with commas.
\keywords{Logistics Performance, LPI, Logistics Decision-Making in Engineering, Feature Aggregation, Exploratory Factor Analysis}
%% A "teaser" image appears between the author and affiliation
%% information and the body of the document, and typically spans the
%% page.

\received{20 February 2007}
\received[revised]{12 March 2009}
\received[accepted]{5 June 2009}

%%
%% This is file `sample-sigconf.tex',
%% generated with the docstrip utility.
%%
%% The original source files were:
%%
%% samples.dtx  (with options: `sigconf')
%% 
%% IMPORTANT NOTICE:
%% 
%% For the copyright see the source file.
%% 
%% Any modified versions of this file must be renamed
%% with new filenames distinct from sample-sigconf.tex.
%% 
%% For distribution of the original source see the terms
%% for copying and modification in the file samples.dtx.
%% 
%% This generated file may be distributed as long as the
%% original source files, as listed above, are part of the
%% same distribution. (The sources need not necessarily be
%% in the same archive or directory.)
%%
%%
%% Commands for TeXCount
%TC:macro \cite [option:text,text]
%TC:macro \citep [option:text,text]
%TC:macro \citet [option:text,text]
%TC:envir table 0 1
%TC:envir table* 0 1
%TC:envir tabular [ignore] word
%TC:envir displaymath 0 word
%TC:envir math 0 word
%TC:envir comment 0 0
%%
%%
%% The first command in your LaTeX source must be the \documentclass
%% command.
%%
%% For submission and review of your manuscript please change the
%% command to \documentclass[manuscript, screen, review]{acmart}.
%%
%% When submitting camera ready or to TAPS, please change the command
%% to \documentclass[sigconf]{acmart} or whichever template is required
%% for your publication.
%%
%%
\documentclass[sigconf]{acmart}

%%
%% \BibTeX command to typeset BibTeX logo in the docs
\AtBeginDocument{%
  \providecommand\BibTeX{{%
    Bib\TeX}}}

%% Rights management information.  This information is sent to you
%% when you complete the rights form.  These commands have SAMPLE
%% values in them; it is your responsibility as an author to replace
%% the commands and values with those provided to you when you
%% complete the rights form.
\setcopyright{acmcopyright}
\copyrightyear{2024}
\acmYear{2024}
\acmDOI{XXXXXXX.XXXXXXX}

%% These commands are for a PROCEEDINGS abstract or paper.
\acmConference[ICoMS 2024]{7th International Conference on Mathematics and Statistics, ICoMS 2024}{June 23--25,
  2024}{Amarante, PT}
%%
%%  Uncomment \acmBooktitle if the title of the proceedings is different
%%  from ``Proceedings of ...''!
%%
%%\acmBooktitle{Woodstock '18: ACM Symposium on Neural Gaze Detection,
%%  June 03--05, 2018, Woodstock, NY}
\acmPrice{15.00}
\acmISBN{978-1-4503-XXXX-X/18/06}


%%
%% Submission ID.
%% Use this when submitting an article to a sponsored event. You'll
%% receive a unique submission ID from the organizers
%% of the event, and this ID should be used as the parameter to this command.
%%\acmSubmissionID{123-A56-BU3}

%%
%% For managing citations, it is recommended to use bibliography
%% files in BibTeX format.
%%
%% You can then either use BibTeX with the ACM-Reference-Format style,
%% or BibLaTeX with the acmnumeric or acmauthoryear sytles, that include
%% support for advanced citation of software artefact from the
%% biblatex-software package, also separately available on CTAN.
%%
%% Look at the sample-*-biblatex.tex files for templates showcasing
%% the biblatex styles.
%%

%%
%% The majority of ACM publications use numbered citations and
%% references.  The command \citestyle{authoryear} switches to the
%% "author year" style.
%%
%% If you are preparing content for an event
%% sponsored by ACM SIGGRAPH, you must use the "author year" style of
%% citations and references.
%% Uncommenting
%% the next command will enable that style.
%%\citestyle{acmauthoryear}


%%
%% end of the preamble, start of the body of the document source.
\begin{document}

%%
%% The "title" command has an optional parameter,
%% allowing the author to define a "short title" to be used in page headers.
\title{Longitudinal Feature Extraction in International Logistics Performance Index}

%%
%% The "author" command and its associated commands are used to define
%% the authors and their affiliations.
%% Of note is the shared affiliation of the first two authors, and the
%% "authornote" and "authornotemark" commands
%% used to denote shared contribution to the research.
\author{Aldina Correia}
\authornote{Both authors contributed equally to this research.}
\email{aic@estg.ipp.pt}
\orcid{0000-0002-4693-4867}
\authornotemark[1]
\affiliation{%
  \institution{CIICESI, ESTG, Instituto Politécnico do Porto}
  \streetaddress{Rua do Curral, Casa do Curral, Margaride}
  \postcode{4610-156}
  \city{Felgueiras}
  \country{Portugal}
}

\author{Diogo Ribeiro}
\email{diogo.ribeiro@mysense.ai}
\orcid{0009-0001-2022-7072}
\affiliation{%
  \institution{MySense.ai}
  \streetaddress{7 Bell Yard}
  \postcode{WC2A 2JR}
  \city{London}
  \country{United Kingdom}}



%%
%% By default, the full list of authors will be used in the page
%% headers. Often, this list is too long, and will overlap
%% other information printed in the page headers. This command allows
%% the author to define a more concise list
%% of authors' names for this purpose.
\renewcommand{\shortauthors}{A. Correia and D. Ribeiro}

%%
%% The abstract is a short summary of the work to be presented in the
%% article.
\begin{abstract}
  The importance of the logistics performance of companies, regions and countries to support decision-making is universally recognised, covering the rationalisation of supply chains, the optimisation of inventory management and promoting global collaboration. 
      
  Efficient logistics integration with innovative technologies is crucial for the prompt delivery of materials and components, increasing the speed and effectiveness of innovation processes and, consequently, the performance of organisations. This study examines the robust correlation structure between Logistics Performance Index (LPI) indicators over several years. 
  
  The LPI assesses global logistical performance by measuring factors such as the quality of commercial and transport infrastructure, the ease of customs procedures and the efficiency of customs clearance, among other aspects that influence the transnational flow of goods.

Our results confirm the LPI as a longitudinal latent variable, characterised by its indicators, which demonstrate remarkable internal consistency. This consistency underpins the reliability of the LPI for assessing global logistics performance.

Recognised as a valuable measure of logistics efficiency, LPI serves as a practical tool in business and politics, guiding strategic decision-making and improving the operational cost-benefit ratio and competitiveness of organisations.
\end{abstract}

%%
%% The code below is generated by the tool at http://dl.acm.org/ccs.cfm.
%% Please copy and paste the code instead of the example below.
%%
\begin{CCSXML}
<ccs2012>
 <concept>
  <concept_id>10010520.10010553.10010562</concept_id>
  <concept_desc>Computer systems organization~Embedded systems</concept_desc>
  <concept_significance>500</concept_significance>
 </concept>
 <concept>
  <concept_id>10010520.10010575.10010755</concept_id>
  <concept_desc>Computer systems organization~Redundancy</concept_desc>
  <concept_significance>300</concept_significance>
 </concept>
 <concept>
  <concept_id>10010520.10010553.10010554</concept_id>
  <concept_desc>Computer systems organization~Robotics</concept_desc>
  <concept_significance>100</concept_significance>
 </concept>
 <concept>
  <concept_id>10003033.10003083.10003095</concept_id>
  <concept_desc>Networks~Network reliability</concept_desc>
  <concept_significance>100</concept_significance>
 </concept>
</ccs2012>
\end{CCSXML}

\ccsdesc[500]{Mathematics of computing~Probability and statistics~Probability and statistics~Statistical paradigms~Dimensionality reduction}
\ccsdesc[300]{Mathematics of computing~Probability and statistics~Probability and statistics~Multivariate statistics}
\ccsdesc[100]{Computing methodologies~Machine learning~Dimensionality reduction and manifold learning}

%%
%% Keywords. The author(s) should pick words that accurately describe
%% the work being presented. Separate the keywords with commas.
\keywords{Logistics Performance, LPI, Logistics Decision-Making, Feature Aggregation, Longitudinal Principal Component Analysis}
%% A "teaser" image appears between the author and affiliation
%% information and the body of the document, and typically spans the
%% page.

\received{20 February 2007}
\received[revised]{12 March 2009}
\received[accepted]{5 June 2009}

%%
%% This command processes the author and affiliation and title
%% information and builds the first part of the formatted document.
\maketitle

\section{Introduction}
Global logistics efficiency is crucial for companies considering international expansion. To assist with this, the World Bank has developed the Logistics Performance Index (LPI), which evaluates countries based on the quality of their trade infrastructure, ease of customs procedures, and efficiency of goods clearance across borders.

The LPI is derived from surveys conducted with global freight forwarders and logistics professionals, capturing various facets of a country's logistics environment. It measures logistics performance through six key dimensions:

Customs Clearance Process: Evaluates the efficiency of customs and border clearance.
Infrastructure Quality: Assesses the quality of trade and transport infrastructure like ports and roads.
Ease of Arranging Shipments: Looks at the ease of organising competitively priced shipments.
Competence and Quality of Logistics Services: Rates the skill and quality of logistics providers.
Tracking and Tracing: Measures the ability to track and trace consignments.
Timeliness of Shipments: Checks the regularity with which shipments meet delivery schedules.
Scores range from 1 to 5, with higher scores indicating superior logistics capabilities. This makes the LPI an invaluable tool for governments, businesses, and researchers to compare logistics performance internationally, spot improvement opportunities, and inform policy decisions.

The 2023 LPI edition introduced new Key Performance Indicators (KPIs) based on big data, which include real-time tracking of shipping containers, air cargo, and parcels, providing a broader perspective on global trade dynamics. These new KPIs complement traditional survey data, offering a fuller picture of logistics performance.

Proposals to enhance the LPI suggest various strategic improvements to better measure and optimize logistics conditions, especially important in scenarios where data collection might be compromised, such as during global disruptions.

The LPI thus not only aids businesses in making informed decisions about international operations but also helps countries in developing policies that foster an efficient logistics framework.

The 2023 edition of the International Logistics Performance Index (LPI) enables comparisons across 139 countries, revealing an overall improvement in global logistics performance over the past decade. Notably, the LPI score increased, with a secondary peak emerging around 3.5 between 2018 and 2023, indicating stronger performance among more countries. However, the reduction in sample size from 160 countries in 2018 to 139 in 2023 complicates direct comparisons, especially at the lower score ranges.

The LPI consistently evaluates a broad spectrum of economies, though the specific countries assessed can vary in each edition due to data availability and varying participation levels. For instance, the 2014 edition did not include data for Portugal.

In 2023, the highest LPI scores were predominantly from high-income economies, with Singapore maintaining its top ranking from previous years with a score of 4.3. Eight of the top twelve scorers were European countries. Conversely, the lowest scores generally came from countries with low to lower-middle incomes facing economic challenges from conflicts, natural disasters, or geographic and economic constraints, affecting their integration into global supply chains.

This paper aims to validate the LPI as a reliable measure of logistical performance by analyzing key characteristics and relationships among its indicators. Recognizing the LPI as a robust metric allows it to inform strategic decision-making in engineering and other sectors, improving operational cost-effectiveness and competitiveness. Additionally, the paper tracks Portugal’s LPI performance from 2007 to 2023, focusing on its changes and implications for logistics strategy.

\section{Data and Results}

The LPI \cite{WB} index is calculated from 6 indicators or six key dimensions \cite{WBreport2016,WBreport2018}:
\begin{enumerate}
    \item \textbf{Customs} Clearance Process: The efficiency of customs and border clearance processes.
    \item \textbf{Infrastructure} Quality: The quality of trade and transport-related infrastructure, including ports, railroads, and roads.
    \item Ease of Arranging \textbf{(International) Shipments}: The ease of arranging competitively priced shipments.
    \item \textbf{Logistics Competence and Quality } of Services: The competence and quality of logistics services, such as transport operators and customs brokers.
    \item \textbf{Timeliness} of Shipments: The frequency with which shipments reach the consignee within the scheduled or expected delivery time.
    \item \textbf{Tracking and Tracing}: The ability to track and trace consignments.
\end{enumerate}

To identify the essential variables that contribute to Countries logistics performance, key indicators selection or extraction techniques can be used, in order to retain the most significant information from the indicators, maximising the variance extracted from the variables in a factor, or several factors.

In \cite{correiaICIE}, to ensure that the extracted characteristics are relevant and informative concerning with 2023  Logistics Performance Index indicators, extraction techniques were used, in particular Exploratory Factor Analysis in the JASP software \cite{JASP}.

The present analysis is grounded in the scores of the LPI indicators over several years, being a longitudinal approach, using R software \cite{R}.

This analysis is based on the scores between 2007 and 2023 LPI \cite{WB} indicators (2007, 2010, 2012, 2014, 2016, 2018, 2023).

Since the Exploratory Factor Analysis (EFA) performed in \cite{correiaICIE} allow to identify a single factor, in this work a Principal Components Analysis was performed over the years available, considering one component. 

Assumptions for this technique  are: (i) sample dimension -- 5 to 10 observations by variable; (ii) normality of the variables; (iii) linearity of the variables; (iv) homocedasticity.

(i) Sample dimension - The sample comprises 139 countries, and there is no missing values. Therefore, the sample size is adequate for the analysis of the six indicators.

(ii) Normality Assessment - For evaluating normality, we can employ the  Mardia's Test of Multivariate Normality \footnote{The statistic for skewness is assumed to be $\chi^2$ distributed and the statistic for kurtosis standard normal.}. This test examines the symmetry and kurtosis of the dataset. The null hypothesis assumes multivariate normality. Since the p-values are greater than 0.05, we do not reject the null hypothesis of the test at a significance level of 5\%. Therefore, there is statistical evidence to suggest that the six variables in our dataset follow a multivariate normal distribution (Appendix \ref{AppendixB} -- Table \ref{tab:mardia'STestOfMultivariateNormality}).




(iii) Linearity -- Pearson's Correlations between Indicators
(Appendix \ref{AppendixB} -- Table \ref{tab:pearson'SCorrelations}) all show statistical significance at a 5\% significance level, furthermore, they are all high, surpassing the threshold of 0.8. This high level of correlation among the indicators suggests a strong interdependence between them.
Consequently, the extraction of factors from these features is not only justified but becomes imperative due to the presence of multicollinearity among the LPI indicators. Multicollinearity indicates a substantial correlation between all dimensions of the LPI, emphasizing the need for factor extraction to address potential redundancy and enhance the robustness of the analysis.

(iv) 
The assumption of homoscedasticity is justifiable based on the descriptive statistics in the Appendix \ref{AppendixB} -- Table \ref{tab:descriptiveStatistics}). In this table, it is evident that the standard deviation values for the indicators are consistently and similar 
 below 1. This uniformity in standard deviations suggests a relatively consistent distribution of the data  around their respective means for each indicator.
%
The observed similarity in standard deviation values reinforces the assumption of homoscedasticity, providing confidence in the reliability of statistical inferences drawn from the data.

After fulfilling the necessary conditions, EFA can be implemented to assess the relationships within the observed variables.

%----- Requires booktabs package -----%


\begin{table}[h]
	\centering
  \caption{Kaiser-Meyer-Olkin Measure of Sample Adequacy and Factor Loadings}
	{
		\begin{tabular}{lrcc}
			\toprule
			Indicator (Score) & MSA & \; \; Factor Loadings \; \; &  Uniqueness\\
\hline
			Customs Score & $0.912$  & $0.950$ & $0.097$ \\
			Infrastructure Score & $0.897$  & $0.960$ & $0.078$ \\
			International Shipments Score & $0.973$  & $0.888$ & $0.212$   \\
			Logistics Competence and Quality Score & $0.932$  & $0.973$ & $0.053$   \\
			Timeliness Score & $0.944$ & $0.922$ & $0.150$ \\
			Tracking and Tracing Score & $0.923$  & $0.964$ & $0.071$ \\
   \hline
   Overall MSA
 & $0.929$  \\
			\bottomrule
		\end{tabular}
	\label{tab:kaiser-Meyer-OlkinTest}
	}
\end{table}

The MSA (Measure of Sample Adequacy) for each variable indicates the suitability of the variable for EFA. It compares the magnitudes of the coefficients of observed correlations with the magnitudes of the coefficients of partial correlations that would be observed if the variable not considered. Measures below 0.5 suggest that it is not appropriate to include that variable in the EFA. The variable with the lowest MSA should be removed one at a time until none are below 0.5.

In this case, all individual MSAs are greater than this threshold, indicating that they are adequate for the analysis. Additionally, the overall MSA or Kaiser-Meyer-Olkin Measure of Sample Adequacy is 0.929, suggesting excellent suitability for applying EFA \cite{maroco2018analise}. 
KMO values closer to 1.0 are consider ideal while values less than 0.5 are unacceptable.

The Bartlett's test of sphericity is employed to test the null hypothesis that the correlation matrix is an identity matrix \cite{beysenbaev2020proposals}. An identity correlation matrix implies that the variables in study are unrelated and is not conducive to factor analysis. For the features to be extracted the $\chi^2$ of the Bartlett's Test is $1776.413 (df=15)$, with $p-value <0.001$, less than 0.05, indicating the rejection of the null hypothesis, for a 5\% significance level, and conditions for the application of EFA.

EFA application  requires a judicious balance between simplicity and comprehensiveness. Striking a compromise between these extremes involves constructing a model with just enough factors to account for significant covariation among measured variables. This compromise entails making decisions about the number of factors to retain in the model for further analysis \cite{beysenbaev2020proposals}. 


The eigenvalues derived from Principal Component Analysis (PCA) have conventionally served as a means to estimate the number of factors to be further explored in a common factor analysis (\cite{carroll1978effect}).

The chi-square goodness-of-fit test has a $\chi^2=62.216$ with $df=9$ and a $p-value <0.001$. This chi-square tests the null hypothesis that the observed data correlation matrix  is a random sample realization from population having correlation matrix equal to the one returned by the extracted factors. That is, the residuals are random noise, sliding to 0 as the sample size
 grows to infinity. 




In determining the number of factors to retain by focusing on eigenvalues exceeding 1, a single factor was extracted, encompassing all LPI indicators explaining 89\% of the variance of these indicators. Within this factor,  Logistics Competence and Quality is the indicator with highest loading ($0.973$), fallowed by Tracking and Tracing, Infrastructure, Customs, Timeliness and International Shipments ($0.888$).  This signifies that all the indicators contribute to the same latent variable, indicating a high level of correlation among them.

The reliability of factors in the context of factor analysis refers to the consistency or stability of the measurement of underlying constructs represented by those factors. There are several ways to assess the reliability of factors, being the commonly used measure the Cronbach's Alpha, a frequentist scale Reliability Statistics. It assesses how well the items within a factor consistently measure the same underlying construct. A higher Cronbach's alpha value (typically above 0.70) indicates greater reliability.
%


\begin{table}[h]
	\centering
  \caption{Frequentist Individual Item Reliability Statistics}
	{
		\begin{tabular}{lc}
			\toprule
			  Item & \; \; If item dropped \; \;  \\
    		    & \; \; Cronbach' $\alpha$ \; \;  \\
\hline
Customs Score	&	0.972	\\
Infrastructure Score	&	0.973	\\
International Shipments Score	&	0.979	\\
Logistics Competence and Quality Score	&	0.970	\\
Timeliness Score	&	0.976	\\
Tracking and Tracing Score	&	0.972	\\
			\bottomrule
					\end{tabular}
	\label{tab:reliability}
	}
\end{table}

The estimate Cronbach's $\alpha$ of the factor obtained in this EFA is $0.978$, indicating excelent reliability (\cite{pestana2008analise}). Additionally, even if an item is dropped, the frequentist individual item of reliability (see Table \ref{tab:reliability}) remain very high. This suggests that each individual item significantly contributes to the overall reliability of the factor, confirming the exceptional internal consistency of the measures employed. This internal consistency is crucial to ensure the reliability and validity of the analysis results. 
Consequently, this provides evidence of the excellent internal consistency of the latent variable.


\section{Conclusion and Future Work}

In this study, a comprehensive examination revealed a strong correlation among all the indicators encompassed within the Logistics Performance Index (LPI). This observation suggests that the LPI can be conceptualized as a latent variable, representing an underlying construct with remarkable internal consistency, as elucidated by its constituent indicators. The high level of correlation among these indicators implies a coherent and unified measurement of the latent variable, reinforcing the reliability and coherence of the LPI as a robust tool for assessing logistics performance. This insight contributes to a nuanced understanding of the interrelationships among the various dimensions captured by the LPI, providing a solid foundation for further analysis and interpretation of logistics performance on both a national and global scale.
Verifiing LPI as a reliable indicator of countries' logistical performance enables its utilization in engineering as a valuable source of practical insights for innovation. It guides optimal strategic decision-making, providing a comprehensive view of a nation's logistical capabilities, including infrastructure, process efficiency, and transportation services. By using LPI as a source of practical knowledge, governments, companies and engineers can identify specific opportunities and challenges related to logistics in various geographic contexts. For example, a country ranking high on the LPI may indicate an environment conducive to the development of new transportation technologies or efficient supply chains. Conversely, a low LPI score may flag areas where engineering interventions are needed to improve transportation infrastructure or optimize logistics processes.

Concerning with Portugal's performance over the period from 2007 to 2023, particularly, focusing on its overall LPI score, has remained consistent since 2007, although its global ranking has undergone fluctuations. In 2007, Portugal ranked within the top 18\%, improving to the top 14\% by 2018. However, in 2023, Portugal dropped to the 38th position, reflecting a dynamic shift. During that year, 27\% of the 139 assessed countries surpassed Portugal's LPI. 

As mentioned in the LPI reports, for example \cite{WBreport2016,WBreport2018}, but namely in the 2023 report \cite{WBreport2023}, several dimensions of the logistic performance can be out of this index, they try to include other KPI in the index, or to complete information with other type of indicators. It can be a good suggestion for future work. Also to develop a logistic performance index for logistic companies and extend these concepts to the industrial level would be very useful for these companies.


As highlighted in the LPI reports, such as those in 2016 and 2018 \cite{WBreport2016, WBreport2018}, and particularly in the 2023 report \cite{WBreport2023}, various dimensions of logistic performance may fall outside the scope of this index. Efforts have been made to incorporate additional Key Performance Indicators into the index or supplement information with other types of indicators. This observation provides valuable insights for potential future research. Furthermore, the suggestion to develop a logistic performance index tailored specifically for logistics companies and extend these concepts to the industrial level could be highly beneficial for such entities.




%%
%% The acknowledgments section is defined using the "acks" environment
%% (and NOT an unnumbered section). This ensures the proper
%% identification of the section in the article metadata, and the
%% consistent spelling of the heading.
\begin{acks}
This work has been supported by national funds through FCT - Fundação para a Ciência e Tecnologia through project UIDB/04728/2020.
\end{acks}

%%
%% The next two lines define the bibliography style to be used, and
%% the bibliography file.
\bibliographystyle{ACM-Reference-Format}
\bibliography{sample-base}


%%
%% If your work has an appendix, this is the place to put it.
\appendix

\section{appendix A}

\subsection{Part One}

Lorem ipsum dolor sit amet, consectetur adipiscing elit. Morbi
malesuada, quam in pulvinar varius, metus nunc fermentum urna, id
sollicitudin purus odio sit amet enim. Aliquam ullamcorper eu ipsum
vel mollis. Curabitur quis dictum nisl. Phasellus vel semper risus, et
lacinia dolor. Integer ultricies commodo sem nec semper.

\subsection{Part Two}

Etiam commodo feugiat nisl pulvinar pellentesque. Etiam auctor sodales
ligula, non varius nibh pulvinar semper. Suspendisse nec lectus non
ipsum convallis congue hendrerit vitae sapien. Donec at laoreet
eros. Vivamus non purus placerat, scelerisque diam eu, cursus
ante. Etiam aliquam tortor auctor efficitur mattis.

\section{Appendix B}

Nam id fermentum dui. Suspendisse sagittis tortor a nulla mollis, in
pulvinar ex pretium. Sed interdum orci quis metus euismod, et sagittis
enim maximus. Vestibulum gravida massa ut felis suscipit
congue. Quisque mattis elit a risus ultrices commodo venenatis eget
dui. Etiam sagittis eleifend elementum.

Nam interdum magna at lectus dignissim, ac dignissim lorem
rhoncus. Maecenas eu arcu ac neque placerat aliquam. Nunc pulvinar
massa et mattis lacinia.

\end{document}
\endinput
%%
%% End of file `sample-sigconf.tex'.

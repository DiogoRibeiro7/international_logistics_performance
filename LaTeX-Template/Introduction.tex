The speed and efficiency with which companies transport goods across international borders, thanks to globalisation, have become fundamental characteristics of their global competitiveness. This encourages the continued development of performance measures such as the
 World Bank's Logistics Performance Index (LPI) \cite{beysenbaev2020}, Global Competitiveness Index (GCI) \cite{wef2020}, Global Enabling Trade Index (GETI) \cite{wef2018} and UNCTAD Liner Shipping Connectivity Index \cite{unctad2020}. Access to these measures quickly, automatically and efficiently is crucial \cite{babayigit2023, shepherd2023}. 
 Globalisation has made the speed and efficiency of international goods transport fundamental to global competitiveness, encouraging the development of measures like the World Bank’s Logistics Performance Index (LPI), Global Competitiveness Index (GCI), Global Enabling Trade Index (GETI), and UNCTAD Liner Shipping Connectivity Index.
%
%
%\textcolor{red}{Frases completas de 1 fonte?}
LPI, measures the performance of countries logistics system and provides a guide for businesses and policymakers interested in global trade and international investments \cite{WBreport2016,WBreport2018,arvis2023,civelek2015}. The LPI allows countries to assess their current logistics-related strengths and weaknesses, identifying the areas in which they need to improve, and benchmark their performance against global standards, to enhance their international trade capabilities \cite{beysenbaev2020,polat2023}.

%Improving the LPI requires a broad and coordinated approach that addresses various factors impacting a logistics system, simplifying and standardising in customs procedures and regulations can significantly reduce the time and cost associated with clearing goods at the border \cite{babayigit2023,beysenbaev2020}. Investment in transport infrastructure, the diversification of international transport options, and the optimisation of transportation methods improve the connectivity, reliability, flexibility, and accessibility of logistics networks \cite{marti2014}. Further investment in human resources, new technologies and innovation is crucial to increase the quality and competence of logistics services. The adoption of digital platforms and tools for better tracking and tracing improves the visibility and security of cargo movements, moreover, reducing variability and uncertainty in logistics operations decreases the probability of failures in delivery \cite{WBreport2018}.

%\textcolor{red}{FONTE???}
%The relevance of the LPI fundamentally derives from its ability to reflect the nuances of a nation's logistics system — a crucial element for trade effectiveness, economic dynamism and comprehensive social progress \cite{rezaei2018, sharif2024}. Nations with high LPI scores often demonstrate more efficient logistics, which not only speeds up the transportation of goods but also strengthens competitiveness and access to markets \cite{worldbank4}. Conversely, low LPI scores can indicate underlying problems, such as high costs, operational delays and significant risks, that can diminish a country's attractiveness in the global trade landscape \cite{arvis2016,arvis2018,arvis2023}. The LPI transcends its function as a simple metric, being crucial for monitoring the effectiveness of logistics reforms and investments, providing policymakers with a tool to assess the impact of strategic changes on national performance \cite{worldbank5}. 

This paper seeks to validate the LPI as a measure of logistical performance by analysing key characteristics and relationships among its indicators. By recognising the LPI as a feasible and robust metric, it can be employed to inform strategic decision-making across business, logistics, and policy sectors, thereby enhancing operational efficiency and competitiveness on a global scale.
For that Principal Component Analysis (PCA) is considered for extracting the essential feature in the data. Additionally, the paper tracks Portugal's LPI performance from 2007 to 2023, focusing on its changes and implications for logistics strategy.
%
The structure considered is divided into four sections. In the first the theme and importance of the LPI index are presented, in the second the main concepts of PCA are highlighted, the third encompasses the data analysis and presents the results obtained, and the fourth consists of the conclusion and future work of the study.
